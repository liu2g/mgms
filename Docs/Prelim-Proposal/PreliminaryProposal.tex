\documentclass[]{article}

\title{Preliminary Project Propsal}
% Define page size and margin size
\usepackage{geometry}
\geometry{
	a4paper,
	total={170mm,257mm},
	left=20mm,
	top=20mm,
}
\usepackage{indentfirst} %to indent firt paragraph
\usepackage{hyperref}
\begin{document}

\maketitle

% Intro
This is the preliminary proposal for a class of 2021 EECS Senior Design (Capstone) Project. The group includes: \textbf{Alan Trester} (Electrical Engineering), \textbf{Sadie Gladden} (Computer Engineering), and \textbf{Zuguang Liu} (Electrical Engineering).


\section*{Project Title: Modular Garden Monitoring System (MGMS)}

	
	
\section*{Project Advisor: Dr. Zachariah Fuchs}


\section*{Team Members \& Technical Skills}
\begin{itemize}
	\item Alan Trester (Electrical Engineer) - C++, Hardware Prototyping and Testing, Technical Writing for Documentation and Grant Proposals, Manufacturing Engineering 
	\item Sadie Gladden (Computer Engineer) - C++, Python, C, Electronics, Software Development, Organization and Communication 
	\item Zuguang Liu (Electrical Engineer) - C/C++, Python, Industrial system design, embedded system design and testing, signal analysis and processing
\end{itemize}

\section*{Problem Statement}
	Lawns and gardens are one of most essential elements for the typical American home. A survey conducted by National Association of Landscape Professionals in 2019 shows that 79 percent of American families value lawns when renting or buying a home, and about one in three Americans garden in their yards multiple times a week\cite{noauthor_new_2019}. \\
	
	Consequently, there is a constantly high demand of water for use in lawns and garden. Per the United States Environmental Protection Agency, about 48 gallons of water is devoted for this use per family per day. Across America, nearly 1/3 of all residential water is used for landscaping irrigation totaling an estimated 9 billion gallons per day\cite{epa_outdoor_nodate}. In a world undergoing climate change with consistent annual water shortages and wildfires in many parts of the world, wasteful water usage is simply unacceptable. \\
	
	The issue of wasteful irrigation is not being addressed as actively as it deserves to be. Although younger generations of Americans tend to value lawns and gardens even more than older generations, more than half of young people failed quizzes on proper landscape care and nearly 7 out of 10 young people wish to see further improvement in their lawns\cite{noauthor_new_2016}. Uninformed (and in turn, irresponsible) lawn care may significantly contribute to the amount of wasteful water usage happening every day.  \\
	
	A 21st-century solution is needed to help new homeowners care for their lawns and gardens in a more informed and effective way while reducing the amount of wasteful water usage that is accounted for by residential lawn care and irrigation.
	
%	TODO: make a nice problem statement
%	
%	Stuff that could be helpful for forming a problem statement
%	\begin{itemize}	
%		\item \textbf{LAWN OWNERS}
%		\item "81\% of americans have a lawn of some sort" 
%		\item "79\% say a lawn is an important factor when buying a home" 
%		\item \href{https://www.businesswire.com/news/home/20190401005679/en/New-Research-Confirms-Americans-Lawns-Green-Spaces}{Link to the above info}
%		\item \textbf{WATER CONSUMPTION}
%		\item "The average American family uses 320 gallons of water per day, about 30 percent of which is devoted to outdoor uses. More than half of that outdoor water is used for watering lawns and gardens. Nationwide, landscape irrigation is estimated to account for nearly one-third of all residential water use, totaling nearly 9 billion gallons per day."
%		\item "In addition, some experts estimate that as much as 50 percent of water used for irrigation is wasted due to evaporation, wind, or runoff caused by inefficient irrigation methods and systems."
%		\item \href{https://19january2017snapshot.epa.gov/www3/watersense/pubs/outdoor.html}{Link to the above info}
%		\item testing \"a
%		\item "Running a typical sprinkler from a standard garden hose (5/8”) for one hour uses about 1,020 gallons of water; if you run it three times per week, that is about 12,240 gallons per month. If you run the sprinkler three times a week during a 90-day billing cycle, you will add about 36,000 gallons of water to your usage."
%	\item \href{https://www.wsscwater.com/customer-service/rates/water-usage.html}{Link for above info}
%	\end{itemize}


\section*{Proposed Solution}


%	We propose a wireless modular gardening system that can monitor garden or yard conditions such as but not exclusively soil temperature, moisture level, and sunlight levels. These modules should be able to wirelessly communicate with each other and a central hub. The central hub will then be able to use the analyze the data from the modules to display graphs of for better understanding of the garden environment. This information could be used to help determine what plants would thrive in that environment, watering amount and frequency, if certain fertilizers might be needed, etc. Other modular goals would be to include a watering module to help with watering the garden based on soil conditions. This module could be expanded to also take into account weather forecast and data to determine if rain would change the amount of water needed. These features would benefit everyday homeowners with a yard, hobbyist gardeners, and possibly industrial farmers. This project has significant value as it can save consumers money on water and help the environment by reducing overwatering and thus wasting water. 
	
	Our proposed solution is a modular garden monitoring system that will be able to provide real-time and historical information about environmental conditions such as soil moisture, temperature, sunlight, humidity, and so on. Simply having this detailed information on-hand will allow homeowners to make more informed decisions on the types of plants to keep in their gardens as well as when and how much to water them. Internet connectivity can take decision making to the next level by being able to crowd-source gardening recommendations and consider local weather predictions for watering. Further system expansions can introduce features such as automatic watering to take work off of homeowners shoulders while reducing human error in the garden care process. Finally, a smart design will allow the system to be flexible and applicable in a variety of scenarios varying with garden size and irrigation needs and even between residential and industrial settings. 
	

\section*{Proposed Preliminary Features}
\noindent Essential features include
\begin{itemize}
	\item Multiple measurement methods for garden environmental conditions, including water, sunlight, temperature, air, etc.
	\item Modular design with a versatile "hub" to accommodate different garden types, sizes, and needs. 
	\item Wireless communication protocol standard in the industry,
	\item Accessible and friendly user interface that visualizes information and suggestions.
	\item Feedback capability to effectively control environmental factors (such as watering)
\end{itemize}
Additional (nice-to-have) features include but not limit to
\begin{itemize}
	\item An algorithm to suggest vegetation based on the user's environment, and
	\item Integration with smart home devices (Alexa, Google Home, etc.).
\end{itemize}
	

%*PLEASE FIX ME*
%\begin{itemize}
%	\item Create the individual garden modules with data collection capailities for attributes such as temperature, moisture level, sunlight level, etc. 
%	\item Create a central hub in order to analyze and format collected data from modules
%	\item Develop a watering module
%	\item Create a unique and easy to use user interface to access data, information, and suggestions
%\end{itemize}
%
\vspace*{\fill}
\bibliography{Refs} 
\bibliographystyle{ieeetr}
\end{document}
