\documentclass[]{article}

\title{Preliminary Project Propsal}
% Define page size and margin size
\usepackage{geometry}
\geometry{
	a4paper,
	total={170mm,257mm},
	left=20mm,
	top=20mm,
}
\usepackage{hyperref}
\begin{document}

\maketitle

% Intro
This is the premilinary proposal for the Senior Design (Capstone) Project for the class of 2021. The group includes three students: \textbf{Alan Trester} (Electrical Engineering), \textbf{Sadie Gladden} (Computer Engineering), and \textbf{Zuguang Liu} (Electrical Engineering).


\section*{Project Title: *PLEASE CHANGE ME*Gardening Stuff and Things}
	
	
\section*{Project Advisor:TBD... initiated conversation with Dr. Zach Fuchs, back up --}

\section*{Team Members \& Technical Skills:}
\begin{itemize}
	\item Alan Trester (Electrical Engineer) - Write Alan things here
	\item Sadie Gladden (Computer Engineer) - C++, Python, C, Electronics, Software Development, Organization and Communication 
	\item Zuguang Liu (Electrical Engineer) - Write Liu Things Here

\end{itemize}

\section*{Problem Statement:}
	TODO: make a nice problem statement
	
	Stuff that could be helpful for forming a problem statement
	\begin{itemize}	
		\item \textbf{LAWN OWNERS}
		\item "81\% of americans have a lawn of some sort" 
		\item "79\% say a lawn is an important factor when buying a home" 
		\item \href{https://www.businesswire.com/news/home/20190401005679/en/New-Research-Confirms-Americans-Lawns-Green-Spaces}{Link to the above info}
		\item \textbf{WATER CONSUMPTION}
		\item "The average American family uses 320 gallons of water per day, about 30 percent of which is devoted to outdoor uses. More than half of that outdoor water is used for watering lawns and gardens. Nationwide, landscape irrigation is estimated to account for nearly one-third of all residential water use, totaling nearly 9 billion gallons per day."
		\item "In addition, some experts estimate that as much as 50 percent of water used for irrigation is wasted due to evaporation, wind, or runoff caused by inefficient irrigation methods and systems."
		\item \href{https://19january2017snapshot.epa.gov/www3/watersense/pubs/outdoor.html}{Link to the above info}
		\item testing \"a
		\item "Running a typical sprinkler from a standard garden hose (5/8”) for one hour uses about 1,020 gallons of water; if you run it three times per week, that is about 12,240 gallons per month. If you run the sprinkler three times a week during a 90-day billing cycle, you will add about 36,000 gallons of water to your usage."
	\item \href{https://www.wsscwater.com/customer-service/rates/water-usage.html}{Link for above info}
	
		
	\end{itemize}
\section*{Proposed Solution:}
	We propose a wireless modular gardening system that can monitor garden or yard conditions such as but not exclusively soil temperature, moisture level, and sunlight levels. These modules should be able to wirelessly communicate with each other and a central hub. The central hub will then be able to use the analyze the data from the modules to display graphs of for better understanding of the garden environment. This information could be used to help determine what plants would thrive in that enrinvonment, watering amount and frequency, if certain fertilizers might be needed, etc. Other modular goals would be to include a watering module to help with watering the garden based on soil conditions. This module could be expanded to also take into account weather forecast and data to determine if rain would change the amount of water needed. These features would benefit eveeryday homeowners with a yard, hobbyist gardeners, and possibly industrial farmers. This project has significant value as it can save consumers money on water and help the environment by reducing overwatering and thus wasting water. 

\section*{Proposed Preliminary Features:}
*PLEASE FIX ME*
\begin{itemize}
	\item Create the individual garden modules with data collection capailities for attributes such as temperature, moisture level, sunlight level, etc. 
	\item Create a central hub in order to analyze and format collected data from modules
	\item Develop a watering module
	\item Create a unique and easy to use user interface to access data, information, and suggestions
\end{itemize}

\section*{References:}
*PROBABLY ADD WHATEVER STAT SOURCES HERE*
\end{document}
