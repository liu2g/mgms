\chapter{Project Problem Statement}
\section{Detailed Problem Description}
	Lawns and gardens are one of most essential elements for the typical American home. A survey conducted by National Association of Landscape Professionals in 2019 shows that 79 percent of American families value lawns when renting or buying a home, and about one in three Americans garden in their yards multiple times a week\cite{noauthor_new_2019}. \\
	
	Consequently, there is a constantly high demand of water for use in lawns and garden. Per the United States Environmental Protection Agency, about 48 gallons of water is devoted for this use per family per day. Across America, nearly 1/3 of all residential water is used for landscaping irrigation totaling an estimated 9 billion gallons per day\cite{epa_outdoor_nodate}. In a world undergoing climate change with consistent annual water shortages and wildfires in many parts of the world, wasteful water usage is simply unacceptable. \\
	
	The issue of wasteful irrigation is not being addressed as actively as it deserves to be. Although younger generations of Americans tend to value lawns and gardens even more than older generations, more than half of young people failed quizzes on proper landscape care and nearly 7 out of 10 young people wish to see further improvement in their lawns\cite{noauthor_new_2016}. Uninformed (and in turn, irresponsible) lawn care may significantly contribute to the amount of wasteful water usage happening every day.  \\
	
	A 21st-century solution is needed to help new homeowners care for their lawns and gardens in a more informed and effective way while reducing the amount of wasteful water usage that is accounted for by residential lawn care and irrigation. 
\section{Preliminary Proposed Solution}
	Our proposed solution is a modular garden monitoring system that will be able to provide real-time and historical information about environmental conditions such as soil moisture, temperature, sunlight, humidity, and so on. Simply having this detailed information on-hand will allow homeowners to make more informed decisions on the types of plants to keep in their gardens as well as when and how much to water them. Internet connectivity can take decision making to the next level by being able to crowd-source gardening recommendations and consider local weather predictions for watering. Further system expansions can introduce features such as automatic watering to take work off of homeowners shoulders while reducing human error in the garden care process. Finally, a smart design will allow the system to be flexible and applicable in a variety of scenarios varying with garden size and irrigation needs and even between residential and industrial settings.