\chapter{Project Problem Statement}
	Lawns have been one of most essential element for a home in America. A survey conducted by National Association of Landscape Professionals in 2019 shows that 79 percent of the families value lawns when renting or buying a home, and about one in three Americans garden in their yards multiple times a week\cite{noauthor_new_2019}. \\
	
	As a consequence, there is a high demand of water for lawns and garden. per United States Environmental Protection Agency, about 48 gallons of water is devoted for this use per family per day. Across America, nearly 1/3 of all residential water is used to landscape irrigation with the estimated amount of 9 billion gallons per day\cite{epa_outdoor_nodate}. \\
	
	In addition, though younger generations of Americans like lawns and gardens more than their parents and grandparents, more than half of the young people failed the quizzes on landscape caring, and nearly 7 out of 10 young people wants improvement in their lawns\cite{noauthor_new_2016}. \\
	
	We would like to propose an automation system to assist garden owners to manage their gardens efficiently as well as save resources such as water and money in the process. 
\section{Preliminary Description}
	We propose a wireless modular gardening system that can monitor garden or yard conditions such as but not exclusively soil temperature, moisture level, and sunlight levels. These modules should be able to wirelessly communicate with each other and a central hub. The central hub will then be able to use the analyze the data from the modules to display graphs of for better understanding of the garden environment. This information could be used to help determine what plants would thrive in that environment, watering amount and frequency, if certain fertilizers might be needed, etc. Other modular goals would be to include a watering module to help with watering the garden based on soil conditions. This module could be expanded to also take into account weather forecast and data to determine if rain would change the amount of water needed. These features would benefit everyday homeowners with a yard, hobbyist gardeners, and possibly industrial farmers. This project has significant value as it can save consumers money on water and help the environment by reducing overwatering and thus wasting water. 
	
\section{Detailed Problem Description}
	\noindent Essential features include
	\begin{itemize}
		\item Multiple measurement methods for garden environment, including water, sunlight, temperature, air, etc.,
		\item Feedback loop system to automatically control the environment,
		\item Modular design and a main controller compatible for multiple modules,
		\item Wireless communication protocol standard in the industry,
		\item Accessible and friendly user interface that visualizes information and suggestions.
	\end{itemize}
	Additional (nice-to-have) features include but not limit to
	\begin{itemize}
		\item An algorithm to suggest vegetation based on the user's environment, and
		\item Integration with smart home devices (Alexa, Google Home, etc.).
	\end{itemize}
